\documentclass[aspectratio=169]{beamer}\usepackage[]{graphicx}\usepackage[]{xcolor}
% maxwidth is the original width if it is less than linewidth
% otherwise use linewidth (to make sure the graphics do not exceed the margin)
\makeatletter
\def\maxwidth{ %
  \ifdim\Gin@nat@width>\linewidth
    \linewidth
  \else
    \Gin@nat@width
  \fi
}
\makeatother

\definecolor{fgcolor}{rgb}{0.345, 0.345, 0.345}
\newcommand{\hlnum}[1]{\textcolor[rgb]{0.686,0.059,0.569}{#1}}%
\newcommand{\hlsng}[1]{\textcolor[rgb]{0.192,0.494,0.8}{#1}}%
\newcommand{\hlcom}[1]{\textcolor[rgb]{0.678,0.584,0.686}{\textit{#1}}}%
\newcommand{\hlopt}[1]{\textcolor[rgb]{0,0,0}{#1}}%
\newcommand{\hldef}[1]{\textcolor[rgb]{0.345,0.345,0.345}{#1}}%
\newcommand{\hlkwa}[1]{\textcolor[rgb]{0.161,0.373,0.58}{\textbf{#1}}}%
\newcommand{\hlkwb}[1]{\textcolor[rgb]{0.69,0.353,0.396}{#1}}%
\newcommand{\hlkwc}[1]{\textcolor[rgb]{0.333,0.667,0.333}{#1}}%
\newcommand{\hlkwd}[1]{\textcolor[rgb]{0.737,0.353,0.396}{\textbf{#1}}}%
\let\hlipl\hlkwb

\usepackage{framed}
\makeatletter
\newenvironment{kframe}{%
 \def\at@end@of@kframe{}%
 \ifinner\ifhmode%
  \def\at@end@of@kframe{\end{minipage}}%
  \begin{minipage}{\columnwidth}%
 \fi\fi%
 \def\FrameCommand##1{\hskip\@totalleftmargin \hskip-\fboxsep
 \colorbox{shadecolor}{##1}\hskip-\fboxsep
     % There is no \\@totalrightmargin, so:
     \hskip-\linewidth \hskip-\@totalleftmargin \hskip\columnwidth}%
 \MakeFramed {\advance\hsize-\width
   \@totalleftmargin\z@ \linewidth\hsize
   \@setminipage}}%
 {\par\unskip\endMakeFramed%
 \at@end@of@kframe}
\makeatother

\definecolor{shadecolor}{rgb}{.97, .97, .97}
\definecolor{messagecolor}{rgb}{0, 0, 0}
\definecolor{warningcolor}{rgb}{1, 0, 1}
\definecolor{errorcolor}{rgb}{1, 0, 0}
\newenvironment{knitrout}{}{} % an empty environment to be redefined in TeX

\usepackage{alltt}
\usetheme{gotham}

% \setbeamercolor{block~title}{
%   
% }

	\usepackage{standalone}
	\usepackage{tikz}
	\usepackage{pgfplots}
	\usepackage{tabularray} % Typeset tabulars and arrays (contains equivalent of longtable, booktabs and dcolumn at least)
		\UseTblrLibrary{booktabs} % to load extra commands from booktabs
	\usepackage{changepage}
	\usepackage{minted}
		\definecolor{codeback}{rgb}{0.90,0.91,0.92}
		\definecolor{codebackdark}{rgb}{0.10,0.11,0.12}

	\newcommand{\famName}[1]{\textsc{#1}}
	\newcommand{\themename}{\textbf{\textsc{Gotham}}}
\IfFileExists{upquote.sty}{\usepackage{upquote}}{}
\begin{document}

\section{Introduction}

\begin{frame}{Synonyms and Definitions}

There are several terms that have been used as synonyms with state space models (SSMs):

\begin{itemize}
  \item Mechanistic model
  \item Hidden Markov model (HMM)
  \item Partially observed Markov process (POMP) model
\end{itemize}

I chose SSM as it is the terminology often preferred by practitioners.
\end{frame}

\begin{frame}{State Space Models (SSM)}
  I Follow the definition used by \citet{durbin12} for a SSM.
  
  \begin{itemize}
  \item Let $\bm{Y}_{1}, \bm{Y}_2, \ldots, \bm{Y}_{N}$ be random variable representing the observed time series. These observations occur at time points $t_1, \ldots, t_N$, and can be vector valued. 
  \item A SSM introduces unobservable (latent) states $\bm{X}_1, \ldots, \bm{X}_N$ at the same observation times. These latent variables are connected to the observations, in a way defined by the model.
  \end{itemize}
  
  I will adopt the shorthand $t_{\seq{1}{N}} = (t_1, \ldots, t_N)$, $\bm{Y}_{\seq{1}{N}} = (\bm{Y}_{1}, \ldots, \bm{Y}_{N})$, and $\bm{X}_{\seq{1}{N}} = (\bm{X}_1, \ldots, \bm{X}_N)$.
  
  When defining a SSM, we often want to include an initial value for the latent states, $\bm{X}_0$.

\end{frame}

\begin{frame}{Likelihood function}

  We assume that the random variables $\bm{Y}_{\seq{1}{N}}$, $\bm{X}_{\seq{0}{N}}$ have a joint probability density $f_{\bm{X}_{\seq{0}{N}}, \bm{Y}_{\seq{1}{N}}}(\bm{x}_{\seq{0}{N}}, \bm{y}_{\seq{1}{N}}; \, \paramVec)$ with respect to some dominating measure (typically Lebesgue or a counting measure), where $\paramVec$ is a parameter vector $\paramVec \in \R^{d_\paramVec}$ that indexes the model.
  
  The difficulty in likelihood-based inference for these models is a result of only $\bm{Y}_{\seq{1}{N}}$ being observable, and thus the likelihood function involves a high-dimensional integral: 
  
\begin{eqnarray}
  \label{eq:likedef}
  \mathcal{L}(\paramVec; \bm{y}^*) = f_{\bm{Y}_{1:N}}\big(\bm{y}_{1:N}^*; \, \paramVec\big) = \int f_{\bm{X}_{\seq{0}{N}}, \bm{Y}_{\seq{1}{N}}}\big(\bm{x}_{\seq{0}{N}}, \bm{y}_{\seq{1}{N}}^*;\, \paramVec\big) \, d\bm{x}_{\seq{0}{N}}.
\end{eqnarray}

\end{frame}

\begin{frame}{POMP models}

A common approach is to treat SSMs as partially observed Markov process (POMP) models. We make the following assumptions: 
\begin{itemize}
  \item We assume that the latent variables are a Markov process
  $$
  f_{\bm{X}_{n} | \bm{X}_{1:n-1}}(\bm{x}_{n} | \bm{x}_{1:n-1}; \, \paramVec) = f_{\bm{X}_{n} | \bm{X}_{n-1}}(\bm{x}_{n} | \bm{x}_{n-1}; \, \paramVec).
  $$
  \item Measurements are conditionally independent
  $$
  f_{\bm{Y}_{n} | \bm{X}_{1:N}, \bm{Y}_{-n}}(\bm{y}_{n} | \bm{x}_{0:N}, \bm{y}_{-n}; \, \paramVec) = f_{\bm{Y}_{n} | \bm{X}_{n}}(\bm{y}_{n} | \bm{x}_{n}; \, \paramVec).
  $$
\end{itemize}
With these assumptions, we can express the joint density as
\begin{eqnarray}
\label{eq:jointLik}
f_{\bm{X}_{0:N}, \bm{Y}_{1:N}}\big(\bm{x}_{0:N}, \bm{y}_{1:N};\, \paramVec\big) = f_{\bm{X}_0}\big(\bm{x}_0;\, \paramVec\big)\prod_{n = 1}^N f_{\bm{X}_n|\bm{X}_{n-1}}\big(\bm{x}_{n}|\bm{x}_{n-1}; \, \paramVec\big)f_{\bm{Y}_n|\bm{X}_{n}}\big(\bm{y}_n|\bm{x}_{n}; \, \paramVec\big).
\end{eqnarray}

\end{frame}

\begin{frame}
\begin{figure}[!ht]
\begin{knitrout}
\definecolor{shadecolor}{rgb}{0.969, 0.969, 0.969}\color{fgcolor}
\includegraphics[width=0.75\maxwidth]{figure/pompDiagram-1} 
\end{knitrout}
\caption{\label{fig:pompDiagram}A flow diagram representing an arbitrary POMP model. Modified figure from SBIED course (King, Ionides).}
\end{figure}

\alert{Each of the SSMs considered in this thesis are POMP models.}

\end{frame}

\begin{frame}{Other synonyms and definitions}
  Other common terms that are sometimes used as synonyms are used for special cases 
  \begin{block}{Mechanistic Model}
    A SSM (or POMP) where the evolution of latent variables is dictated by equations mimicing real-world mechanisms. 
  \end{block}
  
  \begin{block}{Hidden Markov Model (HMM)}
    A SSM (or POMP) where the latent variables take values in a discrete and finite space.
  \end{block}
  
\end{frame}

\begin{frame}{Remaining Chapters and Outline}
  \begin{itemize}
    \item Inference for ARMA models.
    \item Mechanistic models for modeling cholera outbreak in Haiti.
    \item The marginalized panel iterated filter (MPIF) algorithm.
  \end{itemize}
\end{frame}

\end{document}
%EoF
